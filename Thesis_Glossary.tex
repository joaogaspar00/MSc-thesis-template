%%%%%%%%%%%%%%%%%%%%%%%%%%%%%%%%%%%%%%%%%%%%%%%%%%%%%%%%%%%%%%%%%%%%%%%%
%                                                                      %
%     File: Thesis_Glossary.tex                                        %
%     Tex Master: Thesis.tex                                           %
%                                                                      %
%     Author: Andre C. Marta                                           %
%     Last modified : 27 Feb 2024                                      %
%                                                                      %
%%%%%%%%%%%%%%%%%%%%%%%%%%%%%%%%%%%%%%%%%%%%%%%%%%%%%%%%%%%%%%%%%%%%%%%%
%
% The definitions can be placed anywhere in the document body
% and their order is sorted by <key> automatically when
% calling makeindex in the makefile
%
% ----------------------------------------------------------------------
% To create a glossary entry, use the following syntax:
%
% \newglossaryentry{<label>}{name={<key>}, description={<value>}}
%
% where the parameters are:
% <label> is the label of the entry,
% <key> is the acronym to be defined by the glossary entry (in lowercase, preferably)
% <value> is the actual definition of the current term
%
% To produce the desired term in the document, that will be replaced by
% the user-defined in the output, use the following syntax:
%
% \gls{<label>}
%
% ----------------------------------------------------------------------
% To create a acronym entry, use the following syntax:
%
% \newacronym{⟨label⟩}{⟨abbrv⟩}{⟨full⟩}
%
% where the parameters are:
% <label> is the label of the entry,
% <abbrv> is the acronym,
% <full> is the definition of the acronym
%
% To produce the desired term in the document, that will be replaced by
% the user-defined in the output, use one of the following syntaxes:
%
% \acrlong{<label>}
% \acrshort{<label>}
% \acrfull{<label>}
%
% ----------------------------------------------------------------------
% By default, only those entries defined in the main document using the
% commands above will be displayed in the glossary (list of acronyms),
% unless the command \glsaddall is used,
% ----------------------------------------------------------------------

% The order of the definitions below is irrelevant
% since the glossary is automatically ordered alphabetically

\newacronym{iss}{ISS}{International Space Station}

\newacronym{vtc}{VTC}{Vector Thrust Control}
\newacronym{mar}{MAR}{Mid-Air Recovery}


\newacronym{nasa}{NASA}{National Aeronautics and Space Administration}

\newacronym{esa}{ESA}{European Space Agency}
\newacronym{EADS}{EADS Astrium}{European Aeronautic Defence and Space Company Astrium}

\newacronym{armada}{ARMADA}{AutoRotation in Martian Descent And Landing}

\newacronym{edls}{EDLS}{Entry, Descent and Landing System}

\newacronym{rwd}{RWD}{Rotary Wing Decelerator}

\newacronym{bet}{BET}{Blade Element Theory}

\newacronym{leo}{LEO}{Low-Earth Orbit}

\newacronym{ffu}{FFU}{Free Fall Units}

\newacronym{orbc}{ORBC}{On Rocket Board computer}
\newacronym{rxsm}{RXSM}{REXUS Service Module}

\newacronym{dof}{DOF}{Degrees of Freedom}
\newacronym{cg}{CG}{Center of Gravity}

\newacronym{isa}{ISA}{International Standard Atmosphere}
\newacronym{eam}{EAM}{Earth Atmospheric Model}
\newacronym{e-gram}{Earth-GRAM}{Earth Global Reference Atmospheric Model}

\newacronym{cfd}{CFD}{Computational fluid Dynamics}




%\newacronym{⟨label⟩}{⟨abbrv⟩}{⟨full⟩}
%\newacronym{⟨label⟩}{⟨abbrv⟩}{⟨full⟩}

% ----------------------------------------------------------------------
% displays all entries (even those unused with commands \acrlong/short/full)
\glsaddall

% ----------------------------------------------------------------------
% vertical aligment of acronyms' long names
\setlength\LTleft{0pt}
\setlength\LTright{0pt}
\setlength\glsdescwidth{1.0\hsize}
